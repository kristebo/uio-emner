\documentclass[12pt]{beamer}
\usetheme{Szeged}
\usepackage[utf8]{inputenc}
\usepackage[norsk]{babel}
\usepackage{hyperref}
\usepackage{graphicx}
\usepackage{verbatim}
\author{Krister Borge}
\title{3d printing}
\setbeamercovered{transparent} 
\setbeamertemplate{navigation symbols}{} 
\logo{} 
\institute{Institutt for informatikk} 
\date{} 
\subject{INF5922} 
\begin{document}

\begin{frame}
\titlepage
\end{frame}

\begin{frame}
\large{Plan}\\
\small{Nå 3d-printing\\
Uke 41: modellering\\
Uke 42: Laserkutter\\
Uke 43: Arduino\\}
\end{frame}
\begin{frame}{TOC}
\tableofcontents
\end{frame}

\begin{frame}{Om meg}
\section{Om meg}
	Krister Borge - 
	Studerer: Elektronikk og datateknologi
	ansatt på sonen, drift og engasjert i forskjellige prosjekter.
	epost: kristebo@ulrik.uio.no
\end{frame}

\begin{frame}{Hva er 3d-printing}
\section{Hva er 3d-printing}
Hva er 3d-printing:
\begin{enumerate}
\item[•] En metode for å lage ting
\item[•] En metode å overføre en modell fra en 3d-modell på en datamaskin til den virkelige verden.
\item[•] Det finnes mange forskjellige former for 3d-printing
\item[-] FDM, SLA, DLP, SLM, 3DP, SLS, LOM, $\mathsf{EBF^3}$ er noen
\end{enumerate}
\end{frame}

\begin{frame}{Desktop vs Industrielle teknologier}
\section{Desktop vs industrielle teknologier}
Det er et skille mellom de store industrielle maskinene og de små vi ser i dag. skillet ligger i hvilken teknologi som brukes. Det er patent-reglene som bestemmer hvem som kan anvende de forskjellige.
SLA (stereolithografi) og DLP (digital light processing) har lenge vært proprietere og beskyttet av patenter
pulvermaskinene er fremdeles beskyttet av patenter (3dp (inkjet 3d printing), SLS (selective laser sintering) % vil være de første som kommer komerielt tilgjengelig
LOM (lamineringsteknologi) samt $\mathsf{EBF^3}$ (Electron Beam Freeform Fabrication:vaier som sveises/smeltes sammen) er laget for større applikasjoner. 
Vi bruker FDM (Fused deposition modelling)
 
\end{frame}

\begin{frame}{Historien frem til i dag}
\section{Historie frem til i dag}
\href{http://www.engineering.com/3DPrinting/3DPrintingArticles/ArticleID/6262/Infographic-The-History-of-3D-Printing.aspx}{Historien frem til i dag}
\end{frame}

\begin{frame}{Fused Deposition modelling}
\section{Fused Deposition modelling}
FDM er teknologien våre printere bruker:
Disse printerene lager modeller ved bruk av: 
\begin{enumerate}
\item en varm dyse, kalles hot-end
\item noe å bevege denne varme dysen i et plan (kartesiske koordinater)
\item noe å dytte ut materialet med, kalles en ekstruder
\item kontrollelektronikk og steppermotorer
\end{enumerate}
\end{frame}

\begin{frame}{materialer}
\section{Materialer}
PLA - Polylactic acid (er biologisk nedbrytbart og er laget av f. eks stivelse)\\
HPFE - polyethylene,PE (brus flasker)\\
ABS - Acrolonitril  butadiene Styren (LEGO)\\
\end{frame}

\begin{frame}{modellering}
\section{Modellering}
SketchUp \\
Autodesk Inventor/360 \\
Blender \\
\end{frame}

\begin{frame}{SketchUp}
Et helt greit og enkelt modelleringsverktøy.
\href{http://www.sketchup.com/download}{last ned SketchUp}
\end{frame}

\begin{frame}{G-kode}
\section{G-Kode}
G-kode er en industriell standard for CNC maskiner. \\
her leser maskina en tekstfil som inneholder informasjon om hva 3d-printeren skal gjøre.
\end{frame}

\begin{frame}{Slicer}
\section{Slicer}
Sliceren gjør om 3d modellen til noe 3d-printeren forstår.
Ultimakers cura - funker mot de aller fleste maskiner.
\end{frame}



\end{document}